\documentclass[12pt]{article}
\usepackage[utf8]{inputenc}
\usepackage[brazil]{babel}
\usepackage{amsmath, amssymb}
\usepackage{graphicx}
\usepackage{hyperref}
\usepackage{enumitem}
\usepackage{float}

\title{Notas de Aula: Riffle Shuffles e Processos Estocásticos}
\author{Disciplina de Processos Estocásticos}
\date{}

\begin{document}

\maketitle

\section*{Introdução}

Como modelar o ato de embaralhar cartas? Essa pergunta aparentemente simples leva a uma bela aplicação da Teoria das Probabilidades e Processos Estocásticos, com conexões com cadeias de Markov, variação total, o paradoxo do aniversário e o problema do coletor de cupons.

\section{Modelagem do Embaralhamento}

Dado um baralho com $n$ cartas numeradas de $1$ a $n$, podemos embaralhá-lo aplicando uma permutação $\pi \in S_n$ (o grupo das permutações).

O embaralhamento ideal corresponderia a aplicar uma permutação uniformemente aleatória, mas na prática, apenas subconjuntos específicos de $S_n$ são realizados, e com distribuições não uniformes.

\section{Riffle Shuffle (Embaralhamento à la Cassino)}

O \textbf{riffle shuffle} (embaralhamento por entrelaçamento) é uma modelagem realista do embaralhamento feito por crupiês:

\begin{enumerate}[label=(\alph*)]
    \item Divide-se o baralho em duas partes (com corte em $t$).
    \item As cartas são intercaladas uma a uma, respeitando a ordem interna de cada parte.
\end{enumerate}

Formalmente, uma permutação $\pi \in S_n$ é um riffle shuffle se a sequência $(\pi(1), \ldots, \pi(n))$ consiste na intercalação de duas sequências crescentes.

\subsection*{Número de Riffle Shuffles}

Para cada corte em $t$, há $\binom{n}{t}$ formas distintas de intercalar. Portanto, existem $2^n$ possíveis interações (mas apenas $2^n - n$ riffle shuffles distintos).

Para entender quantas permutações diferentes podem ser produzidas por um único \textbf{riffle shuffle}, vamos analisar o processo em detalhes.

\subsection*{Divisão do baralho}

Primeiro, o baralho de $n$ cartas é dividido em duas partes:
\begin{itemize}
    \item A parte da \textbf{mão direita} com $t$ cartas;
    \item A parte da \textbf{mão esquerda} com $n - t$ cartas.
\end{itemize}

Para cada valor possível de $t$ entre $0$ e $n$, há $\binom{n}{t}$ formas diferentes de intercalar as cartas das duas mãos preservando a ordem relativa dentro de cada uma. Ou seja, não embaralhamos dentro de cada metade: apenas as intercalamos.

\subsection*{Intercalações possíveis}

Ao intercalar duas listas ordenadas de tamanhos $t$ e $n - t$, o número de intercalamentos possíveis (também chamados de \textit{interleavings}) é:

\[
\binom{n}{t}
\]

Isso porque estamos escolhendo em quais das $n$ posições finais irão as $t$ cartas da mão direita (as restantes serão ocupadas pelas cartas da mão esquerda).

Como $t$ pode variar de $0$ até $n$, o número total de possíveis intercalações que preservam a ordem interna de cada parte é:

\[
\sum_{t=0}^{n} \binom{n}{t} = 2^n
\]

No entanto, isso é apenas o número total de \textbf{intercalações possíveis}, e não necessariamente de \textbf{riffle shuffles distintos}.

\subsection*{Permutações geradas}

Nem todas as intercalações geram permutações diferentes. Algumas diferentes formas de intercalar podem resultar na mesma permutação final. Além disso, o número de permutações que podem ser obtidas por um riffle shuffle é muito menor do que $n!$ (o total de permutações possíveis).

De fato, o conjunto de permutações obtidas por um riffle shuffle corresponde exatamente às permutações de $n$ elementos que podem ser decompostas como a \textbf{intercalação de duas subsequências crescentes}. Tais permutações têm a propriedade de ter no máximo uma "queda" (ou \textit{descent}), isto é, um índice $i$ tal que $\pi(i) > \pi(i+1)$.

Esse conjunto especial de permutações é bastante restrito: existem exatamente $2^n - n$ permutações desse tipo, chamadas de \textbf{riffle permutations}.

\subsection*{Resumo}

\begin{itemize}
    \item Existem $2^n$ formas diferentes de intercalar duas pilhas preservando ordem interna.
    \item Nem todas essas formas geram permutações distintas.
    \item O número de permutações possíveis por um único riffle shuffle é igual a $2^n - n$.
    \item Isso é muito menor do que $n!$, o número total de permutações possíveis de $n$ cartas.
\end{itemize}

Este fato destaca uma verdade importante: \textbf{um único riffle shuffle não produz uma permutação aleatória uniforme}. É preciso repetir o processo múltiplas vezes para que a distribuição se aproxime da uniforme.

\section{Modelo Probabilístico de Gilbert-Shannon-Reeds}

O modelo clássico e mais estudado para o riffle shuffle foi desenvolvido por \textbf{Edgar Gilbert} e \textbf{Claude Shannon}, e depois formalizado por \textbf{Jim Reeds}. Ele fornece uma descrição estocástica natural para o embaralhamento tipo riffle e nos permite analisá-lo matematicamente.

Esse modelo define uma distribuição de probabilidade $\text{Rif}$ sobre o grupo de permutações $S_n$, onde $n$ é o número de cartas.

Existem três maneiras equivalentes de descrever esse modelo:

\subsection*{1. Definição por permutações permitidas}

O modelo define a distribuição $\text{Rif}(\pi)$ como:

\[
\text{Rif}(\pi) = 
\begin{cases}
\dfrac{n+1}{2^n} & \text{se } \pi = \text{id} \text{ (a identidade)} \\
\dfrac{1}{2^n} & \text{se } \pi \text{ pode ser obtida por riffle shuffle (i.e., intercalação de duas sequências crescentes)} \\
0 & \text{caso contrário}
\end{cases}
\]

Essa definição mostra que:
- A permutação identidade ocorre com probabilidade ligeiramente maior.
- Todas as outras riffle-permutações válidas têm a mesma probabilidade.
- Permutações que não podem surgir de um riffle shuffle têm probabilidade zero.

\subsection*{2. Modelo de corte e intercalamento probabilístico}

Esse é o modelo mais intuitivo para quem já embaralhou cartas fisicamente:

\begin{enumerate}
    \item Corta-se o baralho em dois blocos: o número $t$ de cartas a serem colocadas na mão direita é escolhido com probabilidade
    \[
    \mathbb{P}(t) = \dfrac{1}{2^n} \binom{n}{t}, \quad \text{para } 0 \leq t \leq n.
    \]
    \item As duas metades (de tamanhos $t$ e $n-t$) são intercaladas uma a uma: a próxima carta a ser colocada vem da mão direita com probabilidade proporcional ao número de cartas restantes nela, isto é,
    \[
    \mathbb{P}(\text{direita}) = \dfrac{r}{r+\ell}, \qquad \mathbb{P}(\text{esquerda}) = \dfrac{\ell}{r+\ell},
    \]
    onde $r$ e $\ell$ são os números atuais de cartas nas mãos direita e esquerda, respectivamente.
\end{enumerate}

Esse processo gera uma permutação $\pi \in S_n$ com distribuição $\text{Rif}(\pi)$.

\subsection*{3. Modelo binário (shuffle inverso)}

Neste modelo, cada carta é rotulada independentemente com um bit aleatório $0$ ou $1$, com igual probabilidade:

\[
\mathbb{P}(\text{bit} = 0) = \mathbb{P}(\text{bit} = 1) = \dfrac{1}{2}.
\]

Depois disso:
\begin{itemize}
    \item Todas as cartas com bit $0$ são colocadas no topo do baralho;
    \item As cartas com bit $1$ são colocadas abaixo;
    \item Dentro de cada grupo (0s e 1s), a ordem relativa original das cartas é preservada.
\end{itemize}

Este modelo define uma \textbf{inversa} de um riffle shuffle, pois é equivalente a aplicar uma permutação cuja inversa é uma riffle shuffle válida. Por isso, ele é frequentemente usado para análise teórica, como veremos na próxima seção.

\subsection*{Equivalência dos modelos}

As três descrições acima são probabilisticamente equivalentes, isto é, produzem exatamente a mesma distribuição $\text{Rif}$ sobre $S_n$. A equivalência entre elas pode ser compreendida da seguinte forma:

\begin{itemize}
    \item O modelo 1 é a descrição explícita da distribuição.
    \item O modelo 2 fornece um algoritmo estocástico para gerar permutações com essa distribuição.
    \item O modelo 3 transforma o problema em um contexto mais combinatório (bits aleatórios), útil para prova de teoremas.
\end{itemize}

\subsection*{Vantagens do modelo GSR}

\begin{itemize}
    \item Reflete bem o comportamento de embaralhamentos feitos por humanos (especialmente amadores).
    \item Permite análises precisas, via ferramentas de combinatória e probabilidade.
    \item É compatível com técnicas de stopping time e distância de variação total.
\end{itemize}


\section{Regra de Parada Uniforme Forte}

Uma regra de parada uniforme forte é uma variável aleatória $T$ que representa o tempo de parada de um processo estocástico com a propriedade:

\[
\mathbb{P}[X_k = \pi \mid T = k] = \frac{1}{n!}, \quad \text{para todo } \pi \in S_n.
\]

\subsection*{Regra de Parada para Riffle Shuffles}

Após $k$ embaralhamentos inversos, cada carta adquire uma sequência binária $(b_1, b_2, \ldots, b_k)$. A regra de parada:

\textit{``Pare quando todas as cartas tiverem rótulos binários distintos.''}

\section{Análise via Paradoxo do Aniversário}

Assumindo $k$ embaralhamentos, cada carta recebe um rótulo binário de comprimento $k$, totalizando $2^k$ possibilidades.

O problema se reduz ao paradoxo do aniversário: qual a probabilidade de $n$ objetos aleatórios \emph{não} colidirem em $2^k$ caixas?

\[
\mathbb{P}[T > k] = 1 - \prod_{i=1}^{n-1}\left(1 - \frac{i}{2^k}\right).
\]

\textbf{Teorema 1:} Após $k$ riffle shuffles, a distância de variação total satisfaz:

\[
\| \text{Rif}^{*k} - U \| \leq 1 - \prod_{i=1}^{n-1}\left(1 - \frac{i}{2^k}\right).
\]

\section{Quantas Vezes Embaralhar?}

\begin{itemize}
    \item Para $n = 52$ cartas, temos:
    \[
    \begin{array}{l|l}
        k & \| \text{Rif}^{*k} - U \| \\
        \hline
        1 & 1.000 \\
        5 & 0.952 \\
        7 & 0.334 \\
        10 & 0.043 \\
        12 & 0.028 \\
    \end{array}
    \]
    \item Conclusão prática: \textbf{7 riffle shuffles} são suficientes para tornar o baralho "quase aleatório".
\end{itemize}

\section{Conclusão}

O riffle shuffle é um belo exemplo de aplicação de ferramentas probabilísticas em situações cotidianas. Modelos como o do embaralhamento inverso, regras de parada fortes e conexões com problemas clássicos (aniversário, cupons) revelam a elegância dos processos estocásticos.

\end{document}
